\documentclass{article}
\usepackage[utf8]{inputenc}

\title{Deep Neural Networks for Acoustic Modeling}
\author{}
\date{November 2015}

\usepackage{natbib}
\usepackage{graphicx}

\begin{document}

\maketitle

\section{Project Proposal}
\begin{itemize}
 \item \textbf{Topic of group:} Deep Neural Networks for Acoustic Modeling
 \item \textbf{Group members:} Bajibabu Bollepalli,  Hieu Nguyen,  Rakshith Shetty, and Pieter Smit (Mentor).
 \item \textbf{Work plan:}
   \begin{enumerate}
   \item Frame based phoneme recognition using simple multilayer perceptrons. Quantify performance against no of layers, type of non-linearity used (For eg. sigmoid neurons, Rectified linear units).
   \item Experiments with various input features i.e augmenting two or three neighbour frames of of MFCCs as single frame, and spectrogram features as input to the DNNs.
   \item Compare the results with Gaussian Mixture Model method.
   \item Weight vectors initialization: randomly assigned vs pretraining or other methods (if time permits)
   \item Experiments using different neural network architecutres i.e simple recurrent neural networks (RNNs) and long term short term memory networks (if time permits).
   \end{enumerate}
 \item \textbf{Implementation language:} Python with Theano library
 \item \textbf{Possible papers}
 \citep{Hinton2012, Alex2013, Mohamed2012}
\end{itemize}


\section{Introduction}
The goal of automatic speech recognition (ASR) is to convert a speech signal into corresponding text form. Since five decades the progress of ASR has been improved from recognition of isolated digits to telephone-conversational speech. However, the performance of ASR is still beyond human level. Humans are good at recognizing the speech in different accents, dialects, or pronunciations, and speech in different styles, at different rates, and in different environmental conditions. The variability in speech signals pose a big challange to the convenational ASR system. The system simply fails if you provide new speech signal, which is coming from different souce than in training data. 

Traditional ASR system employs the hidden Markov models (HMMs) to model the speech signal where each acoustic state is modeled by a Gaussian mixture model (GMM). 

% \begin{figure}[h!]
% \centering
% \includegraphics[scale=1.7]{universe.jpg}
% \caption{The Universe}
% \label{fig:univerise}
% \end{figure}

% \section{Conclusion}
% ``I always thought something was fundamentally wrong with the universe'' \citep{adams1995hitchhiker}

\bibliographystyle{plain}
\bibliography{references}
\end{document}